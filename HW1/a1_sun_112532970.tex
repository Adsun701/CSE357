\documentclass[12pt, letterpaper, twoside]{article}
\usepackage[utf8]{inputenc}
\usepackage{amssymb, amsmath}
\usepackage{graphicx}

\title{CSE 357 - HW 1}
\author{Andrew Sun}
\date{September 5 2021}


\begin{document}
	\begin{titlepage}
		\maketitle
	\end{titlepage}
	\begin{enumerate}
		\item \textbf{Dice Dependence (16 points).} You roll two fair
		6-sided dice ($D1$ and $D2$).
		\begin{enumerate}
			\item Which of the following are independent?\\
			$D1=6,D1+D2=5$\\
			$D1=5,D1+D2=6$\\
			$D1=4,D1+D2=7$\\
			None of the above\\
			\textbf{Answer:} $D1=4,D1+D2=7$ are independent events.
			\item Why? (show your work)\\
			\textbf{Answer:} For the first case $D1=6, D1+D2=5$, $D1$ is already
			greater than 5, and $D2$ cannot be negative. Therefore the two events
			here are mutually exclusive and thus cannot be independent.
			
			For the second case $D1=5,D1+D2=6$, the probability that $D1+D2=6$ is
			equal to $\frac{5}{36}$. However, given that $D1$ has to be equal to $5$,
			the probability that $D1+D2=6$ becomes $\frac{1}{6}$ as there is only
			$1$ value $D2$ can take - $1$ - in order to satisfy the equality.
			Therefore since the probability changes when $D1$ is set to 5,
			the two events cannot be independent.
			
			For the third case $D1=4,D1+D2=7$, the probability that $D1+D2=7$ is
			equal to $\frac{6}{36}=\frac{1}{6}$. However, given that $D1$ has to
			be equal to $4$, the probability that $D1+D2=6$ remains the same at
			$\frac{1}{6}$ due to 1 out of six possible values for $D2$ - $3$ - satisfying
			the equality $D1 + D2 = 7$. Therefore since the probability remains the same
			when $D1$ is set to $4$, the two events are independent.
		\end{enumerate}
		\item \textbf{Conditionally Independent Mining? (24 points)}
		Suppose you are given five mines that may or may not contain
		gold, and based on stories you hear of all mines in the area
		you assume there is a 50\% probability that any single mine
		has gold.
		\begin{enumerate}
			\item Under this assumption, what is the probability that
			all five mines have gold?\\
			\textbf{Answer:} The probability that all five mines have gold
			is equal to $(0.5)^5=0.03125$.
			\item Suppose you learn that at least one mine has gold.
			Given only this additional information, what is the
			probability that all five contain gold? (show your work)\\
			\textbf{Answer:} We can use Bayes' theorem: $P(A|B)=P(A \cap B) / P(B)$.
			We let event A be the event where all five mines contain
			gold. We let event B be the event where at least one
			mine has gold. $P(A\cap B) = P(A)$ because $P(A)\subset P(B)$,
			or the set of all events where at least one mine has gold
			includes all the events where all 5 mines have gold.
			We then divide $P(A)$, $0.03125$, by $P(B)=1-0.03125$ (the probability
			that no mines have gold subtracted from 1), to get
			our final answer of $0.03225806451612903$.
			\item Your friend is also given 5 mines (ignore all
			information above). They are all along a road leading
			out of town. You learn that \textunderscore{only} one
			of these mines has gold.
			Your friend knows which mine it is and asks you to guess.
			suggesting he might share some of it if you get it right
			(fun friend!). You guess the mine closest to town,
			what is the probability that you are right?\\
			\textbf{Answer:} The probability that the mine closest
			to town has gold is equal to $\frac{1}{5}=0.2$.
			\item Annoyingly, your friend won’t tell you whether
			you’re right! Instead, he reveals that it’s not the
			mine furthest from town. Should you change your answer?
			Why or why not? (explain probabilistically)\\
			\textbf{Answer:} The answer should \textbf{not} be changed because
			given the removal of the mine furthest from the town,
			the probabilities that each remaining mine is the one that
			contains gold remain the same with respect to each other,
			and thus it would not necessarily be advantageous if
			the answer was changed.\\
			It would only make sense to change
			the answer if it was \textit{explicitly} stated that there was a shift
			in probabilities of the remaining mines such that one had
			a greater probability of containing gold than the others.
		\end{enumerate}
	\end{enumerate}
\end{document}
